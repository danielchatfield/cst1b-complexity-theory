\documentclass{supervision}
\usepackage{course}

% Hi Alex and others,
%
% Thanks for getting in touch. You're right, I'll supervise Complexity Theory
% this term. Would either Tuesday afternoon (after 2pm) or Thursday afternoon
% (from 2.30pm to 5.30pm) work for everyone ?
%
% The exercises for the first supervision are
%
%   Questions 1, 2, 3, 4(only a), 5
%
% from the first exercise sheet:
%
%   http://www.cl.cam.ac.uk/teaching/1213/Complexity/exercise1.pdf
%
% (The sheet for this year is not up yet, but it will probably be the same.
% Don't worry about the exact type of the reduction for Q5(b) -- that will be
% covered later in the course and is not important here. Just use the notion
% of reduction you know from Computation Theory.)
%
% Please submit the answers via email by noon two days before the supervision.
%
% See you next week,
% Jannis

\Supervision{1}
\begin{document}
  \begin{questions}
    \question In the lectures, a proof was sketched showing a $\omega(n log n)$
      lower bound on the complexity of the sorting problem. It was also stated
      that a similar analysis could be used to establish the same bound for the
      Travelling Salesman Problem. Give a detailed sketch of such an argument.
      Can you think of a way to improve the lower bound?

      \begin{solution}
        \emph{Not sure how to adapt the proof to apply to TSP}
      \end{solution}

    \question Consider the language \emph{Unary-Prime} in the one letter
      alphabet $\{ a \}$ defined by:

      \begin{equation*}
        {Unary-Prime} = \{ a^n | n \mbox{ is prime } \}
      \end{equation*}

      Show that this language is in $P$.

      \begin{solution}
        To show that Unary-Prime is in $P$ it is sufficient to demonstrate that
        an algorithm of polynomial running time exists that decides Unary-Prime.

        Such an algorithm is:

        \begin{code}{{}}
          // On input $\{ 0,1 \}^n$:

          for k = 2, ..., n-1:
              if k divides n, reject
          accept
        \end{code}

        The division can be carried in time $\mathcal{O}(n^2)$ for each value
        of k, thus since k ranges from 2 to n-1 the total running time is
        $\mathcal{O}(n^3)$ which is polynomial in time.
      \end{solution}

    \question We say that a propositional formula $\phi$ is in 2CNF if it is a
      conjunction of clauses, each of which contains exactly 2 literals. The
      point of this problem is to show that the satisfiability problem for
      formulas in 2CNF can be solved by a polynomial time algorithm.

      First note that any clause with 2 literals can be written as an
      implication in exactly two ways. For instance $(p \lor \lnot q)$ is
      equivalent to $(q \implies p)$ and $(\lnot p \rightarrow \lnot q)$, and
      $(p \lor q)$ is equivalent to $(\lnot p \rightarrow q)$ and $(\lnot q
      \rightarrow p)$.

      For any formula $\phi$, define the directed graph $G_\phi$ to be the
      graph whose set of vertices is the set of all literals that occur in
      $\phi$, and in which there is an edge from literal $x$ to literal $y$ if,
      and only if, the implication $(x \implies y)$ is equivalent to one of the
      clauses in $\phi$.

      \begin{parts}
        \part If $\phi$ has $n$ variables and $m$ clauses, give an upper bound
          on the number of vertices and edges in $G_\phi$.

          \begin{solution}
            An upper bound on the number of vertices is $2n$ (one for each variable and its negation). The upper bound on the number of edges is $2m$ (two for each clause).
          \end{solution}

        \part Show that $\phi$ is unsatisfiable if, and only if, there is a
          literal $x$ such that there is a path in $G_\phi$ from $x$ to $\lnot
          x$ and a path from $\lnot x$ to $x$.

          \begin{solution}
            Since if $x \rightarrow y$ and $y \rightarrow z$ then $x \rightarrow
            z$, a path from $x$ to $\lnot x$ is equivalent to $\lnot x \lor
            \lnot x$. A path from $\lnot x$ to $x$ is equivalent to $x \lor x$.

            This is trivially unsatisfiable.

            \emph{Not sure about reverse implication}
          \end{solution}

        \part Give an algorithm for verifying that a graph $G_\phi$ satisfies
          the property stated in (b) above. What is the complexity of your
          algorithm?

          \begin{solution}
            Reachability algorithm:

            \begin{enumerate}
              \item mark node $a$, leaving other nodes unmarked and initialise
                set $S$ to $\{ a \}$.
              \item while $S$ is not empty, choose node $i$ in $S$: remove $i$
                from $S$ and for all $j$ such that there is an edge $(i, j)$ and
                $j$ is unmarked, mark $j$ and add to $S$.
              \item if b is marked, accept else reject.
            \end{enumerate}

            The complexity of the algorithm that determines whether there is a
            path between two nodes is $\mathcal{O}(V + E)$, since this might
            need to be computed for each variable, the total complexity is
            $\mathcal{O}(n \times (2n + 2m))$ = $\mathcal{O}(n^2)$.
          \end{solution}

        \part From (c) deduce that there is a polynomial time algorithm for
          testing whether or not a 2CNF propositional formula is satisfiable.

          \begin{solution}
            The algorithm in (c) determines whether a 2CNF formula is
            unsatisfiable, and thus can be used to determine whether it is
            satisfiable as this is simply the opposite. The running time is
            polynomial.
          \end{solution}

        \part Why does this idea not work if we have 3 literals per clause?

          \begin{solution}
            It is no longer possible to convert each clause to a simple
            implication.
          \end{solution}

      \end{parts}

    \question A clause (i.e. a disjunction of literals) is called a \emph{Horn
      clause}, if it contains at most one positive literal. Such a clause can
      be written as an implication: $(x \lor (\lnot y) \lor (\lnot w) \lor
      (\lnot z))$ is equivalent to $((y \land w \land z) \rightarrow x))$.
      HORNSAT is the problem of deciding whether a given Boolean expression
      that is a conjunction of Horn clauses is satisfiable.

      \begin{parts}
        \part Show that there is a polynomial time algorithm for solving
          HORNSAT.

          \small{(Hint: if a variable is the only literal in a clause, it must
          be set to true; if all the negative variables in a clause have been
          set to true, then the positive one must also be set to true.
          Continue this procedure until a contradiction is reached or a
          satisfying truth assignment is found).}

          \begin{solution}
            Each clause corresponds to an implication with a conjunction of zero
            or more positive literals on the left and zero or one positive
            literals on the right.

            Set all variables to false.

            While there exists a clause, $c$ in $\phi$ that is ${false}$ under
            the current truth assignment then if the clause has no variables on
            the right side of the implication then return ``UNSAT'', otherwise
            set the variable on the right side of the implication to ${true}$.
          \end{solution}

        \part In the proof of the NP-completeness of SAT it was shown how to
          construct, for every nondeterministic machine $M$, integer $k$ and
          string $x$ a Boolean expression $\phi$ which is satisfiable if, and
          only if, $M$ accepts $x$ within $nk$ steps. Show that, if $M$ is
          deterministic, then $\phi$ can be chosen to be a conjunction of Horn
          clauses.

          \begin{solution}
            \emph{Not sure}
          \end{solution}

      \end{parts}

    \question In general \emph{k-colourability} is the problem of deciding,
      given a graph $G = (V,E)$, whether there is a colouring $\mathcal{X} : V
      \rightarrow {1,\ldots,k}$ of the vertices such that if $(u, v) \in E$,
      then $\mathcal{X}(u) \neq \mathcal{X}(v)$. That is, adjacent vertices do
      not have the same colour.

      \begin{parts}
        \part Show that there is a polynomial time algorithm for solving
          2-colourability.
          \begin{solution}
            To determine whether a graph can be coloured with 2 colours it is
            sufficient to determine whether the graph contains any odd loops.

            To do this, start at a vertex then travel along the edges, assigning
            either ``odd'' or ``even'' to each vertex alternately. An odd cycle
            can be detected if two vertices with the same index are adjacent.

            The running time of this is $\mathcal{O}(V^2)$.
          \end{solution}

        \part Show that, for each $k$, $k$-colourability is reducible to $k +
          1$-colourability.
          \begin{solution}
            Construct graph $G'$ from $G$ by adding a new vertex adjacent to all
            the old vertices, then $G$ is \emph{$k$-colourable} iff $G'$ is
            \emph{$k+1$-colourable}.
          \end{solution}

          What can you conclude from this about the complexity of
          4-colourability?

          \begin{solution}
            It is NP-complete, since 3-colourability is.
          \end{solution}

      \end{parts}

  \end{questions}
\end{document}
