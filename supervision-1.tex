\documentclass{supervision}
\usepackage{course}

% Hi Alex and others,
%
% Thanks for getting in touch. You're right, I'll supervise Complexity Theory
% this term. Would either Tuesday afternoon (after 2pm) or Thursday afternoon
% (from 2.30pm to 5.30pm) work for everyone ?
%
% The exercises for the first supervision are
%
%   Questions 1, 2, 3, 4(only a), 5
%
% from the first exercise sheet:
%
%   http://www.cl.cam.ac.uk/teaching/1213/Complexity/exercise1.pdf
%
% (The sheet for this year is not up yet, but it will probably be the same.
% Don't worry about the exact type of the reduction for Q5(b) -- that will be
% covered later in the course and is not important here. Just use the notion
% of reduction you know from Computation Theory.)
%
% Please submit the answers via email by noon two days before the supervision.
%
% See you next week,
% Jannis

\Supervision{1}
\begin{document}
  \begin{questions}
    \question In the lecture, a proof was sketched showing a $\omega(n log n)$
      lower bound on the complexity of the sorting problem. It was also stated
      that a similar analysis could be used to establish the same bound for the
      Travelling Salesman Problem. Give a detailed sketch of such an argument.
      Can you think of a way to improve the lower bound?

    \question Consider the language \emph{Unary-Prime} in the one letter
      alphabet {a} defined by:

      % TODO

      Show that this language is in P.

    \question We say that a propositional formula $\phi$ is in 2CNF if it is a
      conjunction of clauses, each of which contains exactly 2 literals. The
      point of this problem is to show that the satisfiability problem for
      formulas in 2CNF can be solved by a polynomial time algorithm.

      First note that any clause with 2 literals can be written as an
      implication in exactly two ways. For instance $(p \lor \lnot q)$ is
      equivalent to $(q \imlies p)$ and $(\lnot p \implies \lnot q)$, and
      $(p \lor q)$ is equivalent to $(\lnot p \implies q)$ and $(\lnot q
      \implies p)$.

      For any formula $\phi$, define the directed graph $G_\phi$ to be the
      graph whose set of vertices is the set of all literals that occur in
      $\phi$, and in which there is an edge from literal $x$ to literal $y$ if,
      and only if, the implication $(x \implies y)$ is equivalent to one of the
      clauses in $\phi$.

      \begin{parts}
        \part If $\phi$ has $n$ variables and $m$ clauses, give an upper bound
          on the number of vertices and edges in $G_\phi$.

        \part Show that $\phi$ is unsatisfiable if, and only if, there is a
          literal $x$ such that there is a path in $G_\phi$ from $x$ to $\lnot
          x$ and a path from $\lnot x$ to $x$.

        \part Give an algorithm for verifying that a graph $G_\phi$ satisfies
          the property stated in (b) above. What is the complexity of your
          algorithm?

        \part From (c) deduce that there is a polynomial time algorithm for
          testing whether or not a 2CNF propositional formula is satisfiable.

        \part Why does this idea not work if we have 3 literals per clause?

      \end{parts}

    \question A clause (i.e. a disjunction of literals) is called a \emph{Horn
      clause}, if it contains at most one positive literal. Such a clause can
      be written as an implication: $(x \lor (\lnot y) \lor (\lnot w) \lor
      (\lnot z))$ is equivalent to $((y \land w \land z) \implies x))$. HORNSAT
      is the problem of deciding whether a given Boolean expression that is a
      conjunction of Horn clauses is satisfiable.

      \begin{parts}
        \part Show that there is a polynomial time algorithm for solving
          HORNSAT.

          \small{(Hint: if a variable is the only literal in a clause, it must
          be set to true; if all the negative variables in a clause have been
          set to true, then the positive one must also be set to true.
          Continue this procedure until a contradiction is reached or a
          satisfying truth assignment is found).}
      \end{parts}

    \question In general \emph{k-colourability} is the problem of deciding,
      given a graph $G = (V,E)$, whether there is a colouring $\mathcal{X} : V
      \implies {1,\ldots,k}$ of the vertices such that if $(u, v) \in E$, then
      $\mathcal{X}(u) \neq \mathcal{X}(v)$. That is, adjacent vertices do not
      have the same colour.

      \begin{parts}
        \part Show that there is a polynomial time algorithm for solving
          2-colourability.

        \part Show that, for each $k$, $k$-colourability is reducible to $k +
          1$-colourability.

          What can you conclude from this about the complexity of
          4-colourability?

      \end{parts}

  \end{questions}
\end{document}
