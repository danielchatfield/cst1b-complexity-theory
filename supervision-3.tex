\documentclass{supervision}
\usepackage{course}

\Supervision{3}
\begin{document}
  \begin{questions}
    \section*{2011 Paper 6 Question 1}
    \question[20] The following is a quotation from an Internet forum on
      cryptography.

      \begin{quotation}
        Cracking RSA is NP-complete so nothing better than brute force is
        possible.
      \end{quotation}

      Your task is to evaluate to what extent (if any) this statement is true.
      For full marks, you will consider the following questions.

      \begin{itemize}
        \item What would it mean, precisely, for ``cracking RSA'' to be
          NP-complete? In particular, what is the decision problem involved and
          what is meant by saying it is NP-complete?

        \item Is the problem, in fact, NP-complete? Why or why not?

        \item What is meant, precisely, by the conclusion, ``nothing better than
          brute force is possible''?

        \item Assuming the premise is correct, i.e. ``cracking RSA is
          NP-complete'', does the conclusion follow? Why or why not?

        \item What is the relationship, more generally, between encryption
          systems and NP-completeness?
      \end{itemize}

    \section*{2008 Paper 6 Question 12}
    \question
      \begin{parts}
        \part[6] Give definitions for the complexity classes ${SPACE}(f)$ (for
          any function $f$); $L$ and $NL$.

        \part Consider the following decision problem:

          \begin{description}
            \item[Reachability] Given a graph $G = (V, E)$ and two distinguished
              vertices $s, t \in V$, does $G$ contain a path from $s$ to $t$.
          \end{description}
          \begin{subparts}
            \subpart[7] Explain why \emph{Reachability} is in the complexity
              class ${NL}$.

            \subpart[7] Show that if \emph{Reachability} were in the class $L$,
              we would have $L = {NL}$.
          \end{subparts}
      \end{parts}

    \section*{2006 Paper 5 Question 12}
    \question[20] Suppose that $f(n)$ is a sensible function (you may like in
      some part of your answer to comment on what the term ``sensible'' might
      mean in this context), then show that the class ${DTIME}(f(n))$ is
      strictly contained in ${DTIME}(f(n)^4)$.

  \end{questions}
\end{document}
