\documentclass{supervision}
\usepackage{course}

\Supervision{}
\begin{document}
  \begin{questions}
    \section*{2007 Paper 5 Question 12}
    \question
      \begin{parts}
        \part[2] Give a precise definition of polynomial-time reductions.
          \begin{solution}
            Given two languages $L_1 \subseteq \Sigma_1^*$, and $L_2 \subseteq
            \Sigma_2^*$, $L_1$ is polynomial-time reducible to $L_2$ if there is
            a polynomial-time computable function $f$ such that for every string
            $x \in \Sigma_1^*$, $f(x) \in L_2$ if, and only if, $x \in L_1$.
          \end{solution}

        \part[3] Give a precise definition of NP-completeness.
          \begin{solution}
            A language is NP-HARD if every language in NP is polynomial-time
            reducible to it.

            A language is NP-COMPLETE if it is NP-HARD and is in NP.
          \end{solution}

        \part Let \emph{Subset Sum} denote the following decision problem:

          Given a set of positive integers $S = \{v_1, \ldots , v_n\}$ and a
          number $t$, determine whether there is a subset of $S$ that sums to
          exactly $t$.

          \begin{subparts}
            \subpart[3] Explain why \emph{Subset Sum} is in NP.
              \begin{solution}
                \emph{Subset Sum} is in NP because it can be decided by a
                nondeterministic machine that non-deterministically guesses
                subsets and verifies whether they sum to $t$.

                The verification process is trivially polynomial time as it is
                simply the summation over the subset, which by virtue of being
                a subset is bounded by the size of the original set.
              \end{solution}

            \subpart[9] Describe a polynomial-time reduction from the problem of
              3-dimensional matching to \emph{Subset Sum}.
              \begin{solution}
                Let $f$ be a function that takes an instance of 3DM ($X$, $Y$,
                $Z$, $M$) and returns an instance of Subset Sum.

                Assign integer labels to each member of sets $X$, $Y$, and $Z$
                such that successive labels are successive powers of 2, starting
                with $2^0 = 1$ and ending with $2^{n-1}$ where $n = |X \cup Y
                \cup Z|$.

                Let $S$ be a new set such that for every triple in $M$, the sum
                of the integer labels in the triple is in $S$.

                Return a tuple of $S$ and $t$ where $t = 2^n - 1$.

                It is clear that $f$ runs in polynomial time since it just
                assigns labels, one for each element in sets $X$, $Y$, and $Z$
                and then for each triple of $M$ sums the 3 elements' labels.

                To show that this is a reduction we show that an instance of 3DM
                has a matching iff $S$ has a subset which sums to $t$.

                $t$ is one less than a power of 2, and thus since each label
                was a power of 2 the only summation of labels that will equal
                $t$ is exactly one of each power of two up to $2^{n-1}$.
                Therefore, if $S$ has a subset which sums to $t$ then there must
                be a subset whose summation is formed from every power of two up
                to $2^{n-1}$. This means that the every element in $X$, $Y$, and
                $Z$ must appear exactly once in the triples that correspond to
                the members of the subset and thus the instance has a matching.

                If an instance has a matching then every element must appear
                exactly once in the triples within the matching. The summation
                over the elements in $S$ that correspond to the triples in the
                matching can trivially be seen to equal $2^n - 1$ since it is
                the sum of successive powers of 2 up to $2^{n-1}$.
              \end{solution}

            \subpart[3] Explain why parts (i) and (ii) above imply that
              \emph{Subset Sum} is NP-complete.
              \begin{solution}
                3DM is NP-HARD. Since every language in NP is polynomial-time
                reducible to 3DM we know that for every langauge $L$ in NP there
                exists a polynomial-time reduction to 3DM $f_L$. Let $g$ be the
                reduction described in part (ii), the composition of $f_L$ and
                $g$ is a polynomial-time reduction from $L$ to Subset Sum and
                thus Subset Sum is NP-HARD.

                In part (i) we showed that Subset Sum was in NP and thus it is
                NP-COMPLETE.
              \end{solution}

          \end{subparts}
      \end{parts}

    \section*{2008 Paper 5 Question 12}
    \question
      \begin{parts}
        \part[3] Give a precise definition of what it means for one decision
          problem to be polynomial-time reducible to another.
          \begin{solution}
            Given two languages $L_1 \subseteq \Sigma_1^*$, and $L_2 \subseteq
            \Sigma_2^*$, $L_1$ is polynomial-time reducible to $L_2$ if there is
            a polynomial-time computable function $f$ such that for every string
            $x \in \Sigma_1^*$, $f(x) \in L_2$ if, and only if, $x \in L_1$.
          \end{solution}

        \part[8] Consider the following two decision problems:

          \begin{description}
            \item[HamCycle] Given a graph $G = (V, E)$ does it contain a cycle
              that visits every vertex exactly once?

            \item[HamPath] Given a graph $G = (V,E)$ and two distinguished
              vertices $s, t \in V$, is there a simple path in $G$ that starts
              at $s$, ends at $t$ and visits every other vertex exactly once?
          \end{description}

          Show that \emph{HamCycle} is polynomial-time reducible to
          \emph{HamPath}.

          \begin{solution}
            Consider an instance of HamCycle $G = (V, E)$, the graph $G$ is
            Hamiltonian if, and only if there exists a cycle that visits every
            vertex exactly once.

            Without loss of generality, we can fix the start/end vertex for the
            cycle since if a Hamiltonian cycle exists from one vertex then
            there must also be ones from ever other vertex as you can treat the
            cycle as a continuous loop and move the start position around it.

            Let's fix the start/end vertex as $v$. The penultimate vertex in
            a cycle has to have an edge between it and $v$ since, without such
            an edge it would not be possible to return to $v$ in one step.

            We can then construct a new graph $G' = (V', E')$ from $G$ where
            $V' = V \cup \{ s, t \}$ and $E'$ contains all edges in $E$ as well
            as an edge from $s$ to $v$ and edges from $t$ to all vertices in $V$
            that share an edge with $v$.

            We now have an instance of HamPath ($G'$, $s$, $t$). If there is a
            Hamiltonian Path then the first vertex after $s$ must be $v$ and
            the last one before $t$ must be a vertex that shares an edge with
            $v$. It can therefore be seen that such a path over $G'$ could be
            converted into a cycle over $G$ by removing $s$ and $t$ from it
            and following the one remaining edge to complete the cycle.
          \end{solution}

        \part The following decision problem is known to be solvable in
          polynomial time:

          \begin{description}
            \item[EulerCycle] Given a graph $G = (V, E)$ does it contain a
              cycle that visits every edge exactly once?
          \end{description}

          What can you conclude about the truth of the following statements?
          Justify your answers.

          \begin{subparts}
            \subpart[3] \emph{EulerCycle} is polynomial-time reducible to
              \emph{HamCycle}.
              \begin{solution}
                HamCycle is NP-complete and therefore every language in NP is
                reducible to it. EulerCycle is in P and is thus in NP since P
                is a subset of NP.

                The statement is therefore true.
              \end{solution}

            \subpart[3] \emph{EulerCycle} is polynomial-time reducible to
              \emph{HamPath}.
              \begin{solution}
                This statement is true. From part (i) EulerCycle is reducible to
                HamCycle and from previous question HamCycle is reducible to
                HamPath and thus by the composition of these reductions
                EulerCycle is reducible to HamPath.
              \end{solution}

            \subpart[3] \emph{HamPath} is polynomial-time reducible to
              \emph{EulerCycle}.
              \begin{solution}
                EulerCycle is in P, if HamPath is polynomial-time reducible to
                it then HamPath must also be in P.

                Since HamPath is NP-HARD this is only true if $P = NP$.
              \end{solution}

          \end{subparts}
      \end{parts}

    \section*{2010 Paper 6 Question 1}
    \question
      \begin{parts}
        \part[4] Give precise definitions of polynomial-time reductions and
          NP-completeness.
          \begin{solution}
            Given two languages $L_1 \subseteq \Sigma_1^*$ and $L_2 \subseteq
            \Sigma_2^*$, $L_1$ is polynomial-time reducible to $L_2$ if there
            is a polynomial-time computable function $f$ such that for every $x
            \in \Sigma_1^*$, $x \in L_1$ if, and only if $f(x) \in L_2$.

            A language is NP-HARD if every language in NP is polynomial-time
            reducible to it. A language is NP-COMPLETE if it is NP-HARD and also
            in NP.
          \end{solution}

        \part[6] Prove that for any language $L$, $L$ is polynomial-time
          reducible to some problem in NP if, and only if, $L$ is in NP.
          \begin{solution}
            If a language is in NP it is reducible to itself.

            For the other direction, I'm going to assume that there is a
            language $L_1$ that is not in NP but is polynomial-time reducible to
            a language in NP $L_2$ and derive a contradiction.

            Let $f$ be the polynomial-time reduction and $M$ be the
            nondeterministic machine that decides $L_2$. Since the running time
            of $f$ is polynomial, the length of $f(x)$ must be polynomial in
            the length of $x$. Since $M$, by definition, is polynomial in the
            length of its input by composition of polynomials is also polynomial
            in the length of $x$.
          \end{solution}

        \part In a simple graph $G = (V, E)$, a set of vertices $X \subseteq V$
          is said to be a \emph{vertex cover} of $G$ if every edge $e \in E$ has
          one endpoint in $X$. A set $X \subseteq V$ is an independent set of
          $G$ if there is no edge between any two vertices in $X$.

          \emph{VERTEX COVER} is defined as the decision problem where, given a
          graph $G = (V, E)$ and a positive integer $k$, we are to determine
          whether $G$ contains a vertex cover with $k$ or fewer vertices.

          \emph{INDEPENDENT SET} is defined as the decision problem where, given
          a graph $G = (V, E)$ and a positive integer $k$, we are to determine
          whether $G$ contains an independent set with $k$ or more vertices.

          \begin{subparts}
            \subpart[2] Show that a set $X$ is a vertex cover of $G$ if, and
              only if, its complement $V \backslash X$ is an independent set of
              $G$.
              \begin{solution}
                Let $Y$ be the complement of $X$, if $Y$ is an independent set
                then there is no edge between two vertices in $Y$. Since every
                edge has to vertices it is clear that each edge must either have
                one vertex in $X$ and one in $Y$ or they are both in $X$. $X$ is
                therefore a vertex cover.

                The reverse implication is similar, if $X$ is a vertex cover
                then at least one of the vertices of every edge must be in $X$
                and thus no edge can have both vertices in $Y$ so $Y$ must be an
                independent set.
              \end{solution}

            \subpart[6] Use this to show that \emph{VERTEX COVER} is
              polynomial-time reducible to \emph{INDEPENDENT SET} and vice
              versa.
              \begin{solution}
                Consider an instance of the VERTEX COVER decision problem, $(V,
                E, k)$. From above if there is a vertex cover $X$ then the
                complement must be an independent set. Since, in this problem,
                we are looking for a vertex cover with $k$ or fewer vertices it
                is sufficient to show that there is an independent set with
                at least $n - k$ vertices (where $n$ is the total number of
                vertices). The problem is therefore reduced to INDEPENDENT SET
                with $n - k$ as the target.

                This function is clearly polynomial-time since it is just
                subtracting $k$ from the number of vertices.

                The reverse argument is identical.
              \end{solution}

            \subpart[2] What can you conclude about the complexity of
              \emph{VERTEX COVER}?
              \begin{solution}
                VERTEX-COVER is reducible to INDEPENDENT-SET and is thus in NP
                as INDEPENDENT-SET is in NP.

                INDEPENDENT-SET is NP-HARD, therefore every problem in NP is
                reducible to it. Since INDEPENDENT-SET is reducible to
                VERTEX-COVER, it follows by composition of reductions that every
                problem in NP is polynomial-time reducible to VERTEX-COVER and
                thus VERTEX-COVER is NP-HARD.

                Since it is NP-HARD and in NP, it is NP-COMPLETE.
              \end{solution}
          \end{subparts}

      \end{parts}

  \end{questions}
\end{document}
